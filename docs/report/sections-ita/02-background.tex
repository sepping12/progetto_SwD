\section{Background}
\label{sec:background}

\subsection{Dependability del Software}

La dependability del software è definita come la capacità di un sistema di fornire servizi affidabili e fidati. Comprende sei attributi principali:

\begin{itemize}
    \item \textbf{Affidabilità (Reliability)}: Continuità del servizio corretto
    \item \textbf{Disponibilità (Availability)}: Prontezza per l'uso corretto
    \item \textbf{Sicurezza (Safety)}: Assenza di conseguenze catastrofiche
    \item \textbf{Security}: Protezione contro attacchi intenzionali
    \item \textbf{Manutenibilità}: Facilità di manutenzione e evoluzione
    \item \textbf{Testabilità}: Facilità di verifica del comportamento
\end{itemize}

\subsection{Mutation Testing}

Il mutation testing è una tecnica di test avanzata che valuta l'efficacia di una test suite introducendo difetti artificiali (mutanti) nel codice. Il \emph{mutation score} è calcolato come:

\begin{equation}
\text{Mutation Score} = \frac{\text{Mutanti Uccisi}}{\text{Mutanti Totali} - \text{NO\_COVERAGE}} \times 100\%
\end{equation}

\begin{table}[htbp]
\centering
\caption{Interpretazione Mutation Score}
\label{tab:mutation-interpretation}
\begin{tabular}{ll}
\toprule
\textbf{Score} & \textbf{Interpretazione} \\
\midrule
< 40\% & Insufficiente - test deboli \\
40-60\% & Sufficiente - miglioramento necessario \\
60-80\% & Buono - qualità adeguata \\
> 80\% & Eccellente - test molto efficaci \\
\bottomrule
\end{tabular}
\end{table}

\subsection{Strumenti di Analisi}

\subsubsection{SonarCloud}

Piattaforma cloud per l'analisi statica del codice che fornisce:
\begin{itemize}
    \item Rating qualità (A-E) per Security, Reliability, Maintainability
    \item Quality Gate configurabili
    \item Integrazione con CI/CD
    \item Metriche: bug, vulnerabilità, code smell, coverage, duplicazione, complessità
\end{itemize}

\subsubsection{JaCoCo (Java Code Coverage)}

Tool per la misurazione della copertura del codice che traccia:
\begin{itemize}
    \item \textbf{Line Coverage}: Percentuale linee eseguite
    \item \textbf{Branch Coverage}: Percentuale branch decisionali eseguiti
    \item \textbf{Method Coverage}: Percentuale metodi invocati
    \item \textbf{Class Coverage}: Percentuale classi utilizzate
\end{itemize}

\subsubsection{PITest (Mutation Testing)}

Framework di mutation testing per Java che:
\begin{itemize}
    \item Genera mutanti usando operatori di mutazione (conditionals, math, return values)
    \item Esegue la test suite per ogni mutante
    \item Classifica i mutanti: KILLED, SURVIVED, NO\_COVERAGE
    \item Supporta esecuzione parallela per prestazioni
\end{itemize}

\subsubsection{JMH (Java Microbenchmark Harness)}

Framework ufficiale OpenJDK per microbenchmarking che offre:
\begin{itemize}
    \item Misurazione throughput e latenza
    \item Warmup automatico della JVM
    \item Fork isolation per risultati accurati
    \item Analisi statistica con deviazione standard
\end{itemize}

\subsubsection{Randoop}

Tool di generazione automatica di test che:
\begin{itemize}
    \item Genera test casuali feedback-directed
    \item Produce regression test e error-revealing test
    \item Supporta output JUnit
    \item Utile per scoprire edge case non considerati
\end{itemize}

\subsubsection{SpotBugs e FindSecBugs}

SpotBugs è il successore di FindBugs per la detection di bug. FindSecBugs è un plugin che aggiunge pattern di sicurezza specifici allineati con OWASP.

\subsubsection{OWASP Dependency-Check}

Tool di analisi delle dipendenze che:
\begin{itemize}
    \item Interroga il database NVD (National Vulnerability Database)
    \item Identifica CVE (Common Vulnerabilities and Exposures)
    \item Calcola CVSS score per prioritizzazione
    \item Supporta soppressione falsi positivi
\end{itemize}

\subsection{Docker e CI/CD}

\subsubsection{Docker Multi-Stage Build}

Pattern di containerizzazione che separa build e runtime:
\begin{itemize}
    \item \textbf{Stage 1 (Builder)}: Compilazione con Maven + JDK
    \item \textbf{Stage 2 (Runtime)}: Esecuzione con solo JRE
    \item \textbf{Vantaggi}: Riduzione dimensione immagine, sicurezza, layer caching
\end{itemize}

\subsubsection{GitHub Actions}

Piattaforma CI/CD nativa GitHub che offre:
\begin{itemize}
    \item Workflow definiti in YAML
    \item Trigger configurabili (push, PR, schedule, manual)
    \item Matrix build per test multi-configurazione
    \item Caching dipendenze per prestazioni
\end{itemize}

\subsection{Spring Boot e Project Lombok}

\subsubsection{Spring Boot}

Framework Java per applicazioni enterprise che implementa:
\begin{itemize}
    \item Convention over configuration
    \item Embedded server (Tomcat, Jetty)
    \item Auto-configuration basata su dipendenze
    \item Spring Data JPA per persistenza
    \item Actuator per monitoring e health check
\end{itemize}

\subsubsection{Project Lombok}

Libreria che riduce boilerplate code tramite annotazioni:
\begin{itemize}
    \item \texttt{@Data}: Genera getter, setter, toString, equals, hashCode
    \item \texttt{@Builder}: Pattern builder per costruzione oggetti
    \item \texttt{@NoArgsConstructor}, \texttt{@AllArgsConstructor}: Costruttori automatici
\end{itemize}

\textbf{Considerazione per Mutation Testing}: Il codice generato da Lombok, essendo standard e provato, è escluso dall'analisi di mutation testing per concentrarsi sulla logica di business effettiva.
