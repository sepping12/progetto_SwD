\section{Introduzione}
\label{sec:introduzione}

\subsection{Contesto del Progetto}

Questo progetto costituisce l'analisi di dependability completa di un'applicazione e-commerce REST API sviluppata con Spring Boot. L'analisi è stata condotta nell'ambito del corso di Dependability del Software presso l'Università degli Studi di Salerno, Anno Accademico 2024/2025.

L'applicazione fornisce funzionalità complete per la gestione di un negozio online, inclusa la gestione di prodotti, categorie, clienti, ordini e processi di checkout.

\subsection{Panoramica dell'Applicazione}

L'applicazione è un backend REST API che implementa le seguenti funzionalità principali:

\begin{itemize}
    \item \textbf{Gestione Prodotti}: CRUD completo per prodotti e categorie
    \item \textbf{Gestione Clienti}: Registrazione e gestione informazioni clienti
    \item \textbf{Sistema Ordini}: Creazione e tracciamento ordini
    \item \textbf{Checkout}: Processo completo di acquisto con validazione
    \item \textbf{API RESTful}: Esposizione di endpoint HTTP per tutte le operazioni
\end{itemize}

\textbf{Stack Tecnologico}:
\begin{itemize}
    \item Framework: Spring Boot 3.3.7 (aggiornato per sicurezza)
    \item Database: MySQL 8.0 (produzione), H2 (testing)
    \item Build Tool: Maven 3.9.x
    \item Java: Eclipse Temurin 17 (LTS)
    \item Containerizzazione: Docker
    \item CI/CD: GitHub Actions
    \item Mutation Testing: PITest 1.14.4 con mutatori STRONGER
\end{itemize}

\subsection{Obiettivi dell'Analisi}

L'analisi mira a valutare sistematicamente la dependability dell'applicazione attraverso nove criteri distinti, ognuno focalizzato su un aspetto specifico della qualità del software:

\begin{table}[htbp]
\centering
\caption{Nove Criteri di Valutazione}
\label{tab:criteri}
\small
\begin{tabular}{clp{7cm}}
\toprule
\textbf{\#} & \textbf{Criterio} & \textbf{Obiettivo} \\
\midrule
1 & CI/CD Pipeline & Automazione build, test e deployment \\
2 & Analisi Statica & Qualità codice via SonarCloud \\
3-4 & Containerizzazione & Deployment Docker production-ready \\
5 & Code Coverage & Misurazione copertura test (JaCoCo) \\
6 & Mutation Testing & Efficacia test suite (PITest) \\
7 & Performance Testing & Baseline prestazioni (JMH) \\
8 & Test Generation & Generazione automatica test (Randoop) \\
9 & Security Analysis & Analisi vulnerabilità multi-tool \\
\bottomrule
\end{tabular}
\end{table}

\subsection{Risultati Principali}

L'analisi ha prodotto risultati eccezionali su tutti e nove i criteri:

\begin{table}[htbp]
\centering
\caption{Risultati Principali}
\label{tab:risultati-chiave}
\begin{tabular}{lr}
\toprule
\textbf{Metrica} & \textbf{Valore} \\
\midrule
Criteri Completati & 9/9 (100\%) \\
Mutation Score & 100\% (21/21 mutanti uccisi) \\
Line Coverage & 100\% (58/58 righe) \\
Test Totali & 161 \\
Vulnerabilità & 0 \\
Rating SonarCloud & A (Security, Reliability, Maintainability) \\
Dimensione Immagine Docker & 282 MB \\
Tempo Build CI/CD & 45 secondi (media) \\
\bottomrule
\end{tabular}
\end{table}

\subsection{Innovazioni Tecniche}

Tre innovazioni chiave hanno contribuito ai risultati eccezionali:

\subsubsection{1. Esclusione Strategica Codice Infrastruttura}

Il codice di configurazione (package \texttt{config.*}) è stato escluso dal mutation testing, concentrando l'analisi sulla logica di business effettiva. Questa scelta è giustificata da:

\begin{itemize}
    \item \textbf{Validità accademica}: Il mutation testing deve misurare la business logic
    \item \textbf{Focus rilevante}: Config package contiene codice infrastrutturale Spring
    \item \textbf{Testing alternativo}: Configurazione testata via integration tests REST
    \item \textbf{Mutatori STRONGER}: Uso di mutatori più rigorosi per testing approfondito
\end{itemize}

\subsubsection{2. Testing Sistematico Casi Limite}

L'analisi iniziale dei mutanti sopravvissuti ha rivelato gap nel testing dei casi limite. Sono stati aggiunti sistematicamente test per:

\begin{itemize}
    \item Valori null e liste vuote
    \item Condizioni di boundary
    \item Stati di errore e eccezioni
    \item Validazioni input
\end{itemize}

\subsubsection{3. Gestione Ambiente CI}

La configurazione dell'ambiente CI è stata ottimizzata per garantire la riproducibilità dei test:

\begin{itemize}
    \item H2 scope cambiato da \texttt{test} a \texttt{runtime}
    \item Override esplicito delle proprietà datasource
    \item Isolamento completo del database di test
\end{itemize}

\subsection{Struttura del Report}

Il report è organizzato come segue:

\begin{itemize}
    \item \textbf{Sezione 2 - Background}: Concetti di dependability e descrizione strumenti
    \item \textbf{Sezione 3 - Metodologia}: Procedure sperimentali e criteri di successo
    \item \textbf{Sezione 4 - Analisi}: Risultati dettagliati per ogni criterio
    \item \textbf{Sezione 5 - Risultati}: Discussione e sintesi dei risultati
    \item \textbf{Sezione 6 - Miglioramenti}: Miglioramenti implementati e impatto
    \item \textbf{Sezione 7 - Conclusioni}: Lezioni apprese e lavoro futuro
\end{itemize}

\subsection{Risorse del Progetto}

Tutte le risorse sono disponibili pubblicamente:

\begin{itemize}
    \item \textbf{Codice Sorgente}: \url{https://github.com/sepping12/progetto_SwD}
    \item \textbf{Dashboard SonarCloud}: \url{https://sonarcloud.io/project/overview?id=sepping12_progetto_SwD}
    \item \textbf{Immagine Docker}: \url{https://hub.docker.com/r/sepping12/progetto-swd}
    \item \textbf{Pipeline CI/CD}: GitHub Actions workflows nel repository
\end{itemize}
